\documentclass{article}
\usepackage[a4paper]{geometry}
\usepackage{amsmath}
\geometry{verbose,tmargin=2cm,bmargin=2cm,lmargin=2cm,rmargin=2cm}

\begin{document}

\title{Nonlinear stochastic differential equation generating $1/f$ noise}

\author{Julius Ruseckas}

\date{\today{}}

\maketitle

\section{Nonlinear stochastic differential equations with multiplicative noise}

In Refs.~\cite{kaulakys2004,kaulakys2006} nonlinear stochastic differential
equations (SDEs) of the form
\begin{equation}
dx_{t}=\sigma^{2}\left(\eta-\frac{\lambda}{2}\right)x_{t}^{2\eta-1}dt
+\sigma x_{t}^{\eta}dW_{t}\label{eq:sde-1}
\end{equation}
have been proposed. Here $W_{t}$ is a standard Wiener process (the
Brownian motion), $\eta$ is the power-law exponent of multiplicative
noise and $\sigma$ is the amplitude of the noise. In order to avoid
the divergence of the steady-state probability density function (PDF),
Eq.~(\ref{eq:sde-1}) should be considered together with appropriate
restriction of the diffusion of the stochastic variable $x$. Here
we investigate the SDE with the exponential restriction of diffusion
at $x=x_{\mathrm{min}}$
\begin{equation}
dx_{t}=\sigma^{2}\left(\eta-\frac{\lambda}{2}+\frac{m}{2}
\left(\frac{x_{\mathrm{min}}}{x}\right)^{m}\right)x_{t}^{2\eta-1}dt
+\sigma x_{t}^{\eta}dW_{t}\,.
\label{eq:SDE-our}
\end{equation}
Equation (\ref{eq:SDE-our}) has stationary probability distribution
function (PDF) of the form
\begin{equation}
P_{0}(x)\sim x^{-\lambda}
\exp\left(-\left(\frac{x_{\mathrm{min}}}{x}\right)^{m}\right)\,.
\end{equation}
Based on scaling consideration \cite{ruseckas2014} it is predicted
that the power spectral density (PSD) of the signal $x_{t}$ has $f^{-\beta}$
behaviour in a wide range of frequencies, with 
\begin{equation}
\beta=1+\frac{\lambda-3}{2(\eta-1)}\,.
\end{equation}
Using the parameters $\lambda=2\eta$ and $m=2\eta-2$, equation
(\ref{eq:SDE-our}) takes the form of the \emph{Constant Elasticity of Variance}
(CEV) process:
\begin{equation}
dx_{t}=\sigma^{2}(\eta-1)x_{\mathrm{min}}^{2(\eta-1)}x_{t}dt
+\sigma x_{t}^{\eta}dW_{t}\,.
\end{equation}
If $\lambda=3$ then SDE (\ref{eq:SDE-our}) gives $1/f$ spectrum.
Thus the CEV process has $1/f$ spectrum when $\eta=\frac{3}{2}$:
\begin{equation}
dx_{t}=\mu x_{t}dt+\sigma x_{t}^{\frac{3}{2}}dW_{t}\,.\label{eq:SDE-CEV}
\end{equation}
Here
\begin{equation}
\mu=\frac{1}{2}\sigma^{2}x_{\mathrm{min}}\,.
\end{equation}


\subsection{Derivation of the analytical expression for the power spectral density}

The analytical expression for the transition probability $P_{x}(x',t|x,0)$
(the conditional probability that at time $t$ the signal has value
$x$ with the condition that at time $t=0$ the signal had the value
$x_{0}$) of the CEV process is \cite{kazakevicius2016}
\begin{equation}
P_{x}(x,t|x_{0},0)=\frac{x_{\mathrm{min}}}{(1-e^{-\mu t})}
\sqrt{\frac{x_{0}}{x^{5}}}\exp\left(\frac{1}{2}\mu t
-\frac{x_{\mathrm{min}}}{(1-e^{-\mu t})}\left(\frac{1}{x}
+\frac{1}{x_{0}}e^{-\mu t}\right)\right)
I_{1}\left(\frac{x_{\mathrm{min}}}{\sinh\left(\frac{1}{2}\mu t\right)}
\frac{1}{\sqrt{x_{0}x}}\right)\,.
\label{eq:trans}
\end{equation}
Here $I_{z}$ is the modified Bessel function with index $z$. The
steady-state PDF has the form
\begin{equation}
P_{0}(x)=\frac{x_{\mathrm{min}}^{2}}{x^{3}}
\exp\left(-\frac{x_{\mathrm{min}}}{x}\right)\,.
\end{equation}
The average of the signal is
\begin{equation}
\bar{x}=\int_{0}^{\infty}xP_{0}(x)dx=x_{\mathrm{min}}\,.
\end{equation}
The autocorrelation function can be calculated using the expression
\begin{equation}
C(t)=\int dx\int dx'\,(x-\bar{x})(x'-\bar{x})P_{0}(x)P_{x}(x',t|x,0)\,.
\end{equation}
Using Eq.~(\ref{eq:trans}) and performing the integration we obtain
the autocorrelation function
\begin{equation}
C(t)=x_{\mathrm{min}}^{2}\left[-e^{\mu t}
\ln\left(1-e^{-\mu t}\right)-1\right]\,.
\label{eq:autocorr}
\end{equation}
When $\mu t\ll1$ we get
\begin{equation}
C(t)\approx-x_{\mathrm{min}}^{2}-x_{\mathrm{min}}^{2}\ln(\mu t)\,.
\end{equation}
Similar expansion has been obtained for the autocorrelation function
in the case of $1/f$ spectrum.

According to Wiener-Khintchine relations, the power spectral density
is connected with the autocorrelation function via the transformation
\begin{equation}
S(f)=2\int_{-\infty}^{\infty}C(t)e^{i\omega t}dt=4\int_{0}^{\infty}C(t)
\cos(\omega t)dt\,,
\label{eq:wk}
\end{equation}
where $\omega=2\pi f$ . Using Eq.~(\ref{eq:autocorr}) for the autocorrelation
function we get the following expression for the power spectral density:
\begin{equation}
S(f)=2x_{\mathrm{min}}^{2}\left[\frac{-\gamma
-\psi\left(-i\frac{\omega}{\mu}\right)}{\mu+i\omega}
+\frac{-\gamma-\psi\left(i\frac{\omega}{\mu}\right)}{\mu-i\omega}\right]\,,
\end{equation}
where $\gamma\approx0.577216$ is the Euler's constant and
$\psi(z)=\Gamma^{\prime}(z)/\Gamma(z)$ is the digamma function. When
$\omega\gg\mu$ then the power spectral density is
\begin{equation}
S(f)\approx\frac{2\pi x_{\mathrm{min}}^{2}}{\omega}\,.
\end{equation}


\section{Method of numerical solution}

Method of numerical solution with a variable time step is described
in Ref.~\cite{ruseckas2016}. For the numerical solution, we use
Euler-Marujama approximation, transforming differential equations
to difference equations. If the time step is $\Delta t=h$ then the
difference equations, corresponding to Eq.~(\ref{eq:SDE-CEV}) are
\begin{align}
x_{k+1} = & x_{k}+\mu x_{k}h+\sigma x_{k}^{\frac{3}{2}}\sqrt{h}\varepsilon_{k}\\
t_{k+1} = & t_{k}+h
\end{align}
Here $\varepsilon_{k}$ are normally distributed uncorrelated random
variables with a zero expectation and unit variance. Variable time
step of integration 
\[
h_{k}=\frac{\kappa^{2}}{\sigma^{2}x_{k}}
\]
leads to the equations
\begin{align}
x_{k+1} = & x_{k}+\frac{1}{2}\kappa^{2}x_{\mathrm{min}}+\kappa x_{k}\varepsilon_{k}\\
t_{k+1} = & t_{k}+\frac{\kappa^{2}}{\sigma^{2}x_{k}}
\end{align}
Here $\kappa\ll1$ is a small parameter.

\begin{thebibliography}{1}
\bibitem{kaulakys2004}B.~Kaulakys and J.~Ruseckas, \textit{Stochastic
nonlinear differential equation generating 1/f noise}, Phys.~Rev.~E
\textbf{70}, 020101 (2004).

\bibitem{kaulakys2006}B.~Kaulakys, J.~Ruseckas, V.~Gontis and
M.~Alaburda, \textit{Nonlinear stochastic models of 1/f noise and power-law
distributions}, Physica A \textbf{365}, 217\textendash 221 (2006).

\bibitem{ruseckas2014}J.~Ruseckas and B.~Kaulakys, \textit{Scaling
properties of signals as origin of 1/f noise}, J.~Stat.~Mech.\ \textbf{2014},
P06005 (2014).

\bibitem{kazakevicius2016}R.~Kazakevi\v{c}ius and J.~Ruseckas,
\textit{Influence of external potentials on heterogeneous diffusion
processes}, Phys.~Rev.~E \textbf{94}, 032109 (2016).

\bibitem{ruseckas2016}J.~Ruseckas, R.~Kazakevi\v{c}ius and B.~Kaulakys,
\textit{1/f noise from point process and time-subordinated Langevin
equations}, J.~Stat.~Mech.\ \textbf{2016}, 054022 (2016).
\end{thebibliography}

\end{document}
